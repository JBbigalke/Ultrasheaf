%============================================================
% Definability and Hilbert Space Constructions over J_2
% Root file for arXiv submission
% Compilation: pdfLaTeX + BibTeX
%============================================================
\documentclass[11pt,a4paper]{article}

%============================================================
% preamble-base.tex
% Base preamble file
% Purpose:
%   Core packages and theorem environments
% Engine:
%   pdfLaTeX
%============================================================


% -------------------------------------------------
% Fonts and typography
% -------------------------------------------------
\usepackage[T1]{fontenc}
\usepackage{lmodern}
\usepackage{microtype}

% -------------------------------------------------
% Page layout
% -------------------------------------------------
\usepackage[margin=2.2cm]{geometry}

% -------------------------------------------------
% Mathematics
% -------------------------------------------------
\usepackage{amsmath,amssymb,amsthm}
\usepackage{mathtools}

% -------------------------------------------------
% Lists and tables
% -------------------------------------------------
\usepackage{enumitem}
\usepackage{booktabs}
\usepackage{array}

% -------------------------------------------------
% References and links
% -------------------------------------------------
\usepackage[unicode=true]{hyperref}
\hypersetup{
  colorlinks=true,
  linkcolor=blue,
  citecolor=blue,
  urlcolor=blue,
  pdfencoding=auto
}
\usepackage[nameinlink]{cleveref}

% -------------------------------------------------
% Theorem environments
% -------------------------------------------------
\theoremstyle{plain}
\newtheorem{theorem}{Theorem}[section]
\newtheorem{lemma}[theorem]{Lemma}
\newtheorem{proposition}[theorem]{Proposition}
\newtheorem{corollary}[theorem]{Corollary}

\theoremstyle{definition}
\newtheorem{definition}[theorem]{Definition}

\theoremstyle{remark}
\newtheorem{remark}[theorem]{Remark}


% -------------------------------------------------
% Title information
% -------------------------------------------------
\title{Ultrafilter Aggregation of $\Delta_0$-Definable Fragments and Induced Hilbert Structures}
\author{J\"urgen Bigalke\thanks{j.bigalke@juergen-bigalke.de}}
\date{}

\begin{document}

\maketitle
\thispagestyle{empty}

\begin{abstract}
\noindent
We study bounded $\Delta_0$-definability over the Jensen level 
$J_2 = \mathrm{rud}(\mathrm{HF})$ and analyse an ultrafilter-based 
aggregation mechanism transforming locally definable data into a 
canonical global comparison structure.

Countable families of locally $\Delta_0$-definable fragments are 
aggregated via a fixed non-principal ultrafilter into second-order 
objects (Ultrasheaves). $\Delta_0$-definable numerical comparison 
invariants induce, under aggregation, a well-defined global 
positive semidefinite kernel uniquely determined by the local data.

Under a finitary positivity assumption, the associated kernel 
construction yields a Hilbert structure canonical up to unitary 
equivalence. The resulting quadratic and factorisation properties 
are shown to arise purely from stabilisation phenomena governed 
by bounded definability over $J_2$, without additional analytic 
or topological assumptions.
\end{abstract}


%============================================================
% Document metadata
%============================================================
\vspace{0.8em}
\noindent\textbf{Keywords:}
$\Delta_0$--definability,
Jensen hierarchy,
ultrafilter aggregation,
kernel methods,
Hilbert--space representations

\vspace{0.8em}
\noindent\textbf{MSC 2020:}
03C62; 03E15


\pagenumbering{arabic}
\setcounter{page}{1}

%============================================================
% Main text
%============================================================



%============================================================
\section{Introduction}
\label{sec:introduction}
%============================================================

Initial segments of the Jensen hierarchy provide a canonical framework for
the analysis of definability under severely bounded logical resources.
Among these, the level
\[
J_2=\mathrm{rud}(\mathrm{HF})
\]
occupies a distinguished position: it is a transitive structure closed
under all rudimentary functions, yet sufficiently low to admit a detailed
fine--structural analysis.
Classical references include \cite{Jensen1972,Devlin1984}, with a focused
systematic study of \(J_2\) given in \cite{Weaver2005}.

Analytic representation spaces such as Banach or Hilbert spaces are
traditionally introduced by enriching a set--theoretic framework with
metric, topological, or measure--theoretic structure.
From a fine--structural and definability--theoretic perspective, this
raises a natural structural question: to what extent are such
representations already implicit in the internal resources of set theory
itself, without appeal to external analytic primitives?

The present article addresses this question at a purely structural level.
We show that Hilbert--space representations arise canonically from bounded
\(\Delta_0\)--definable data once local definability is combined with a
single global aggregation principle.
The analysis is carried out concretely over \(J_2\), which serves as a
minimal environment in which nontrivial global structure can emerge from
purely local definability constraints.
While the construction is not intrinsically tied to \(J_2\), this level is
chosen for reasons of canonicity and minimality: it is the smallest
natural domain supporting the required stability properties without
additional axiomatic assumptions.

The central mechanism is an ultrafilter--based aggregation of locally
definable data.
We consider countable families of locally \(\Delta_0\)--definable subsets
of \(J_2\), referred to as \emph{local fragments}.
Aggregating such families along a fixed non--principal ultrafilter yields
global second--order objects, termed \emph{Ultrasheaves}.
The ultrafilter is treated throughout as an externally fixed parameter,
characterised at the metatheoretic level by its standard axioms, and is not
assumed to be definable over \((J_2,\in)\).

Locally \(\Delta_0\)--definable numerical comparison invariants between
fragments induce, under ultrafilter aggregation, a positive semidefinite
kernel on the class of Ultrasheaves.
Assuming only an explicit finitary positivity condition on this local
comparison data, the associated global kernel is uniquely determined.
By the standard kernel--to--Hilbert construction, such a kernel admits a
Hilbert--space representation that is canonical in the
representation--theoretic sense, that is, unique up to unitary
equivalence.
Numerical and geometric structure enter only at the level of
representation; no topological, measure--theoretic, or probabilistic
primitives are presupposed at the level of definability.
All constructions are carried out within bounded \(\Delta_0\)--definability
over \((J_2,\in)\), rudimentary closure, and standard ultrafilter
techniques.
Positivity of the local comparison invariant is taken as explicit
structural input: it is the minimal finitary condition ensuring that the
aggregated comparison data determine a positive semidefinite kernel and
hence admit a Hilbert--space representation.

The novelty of the present analysis therefore does not lie in the
kernel--to--Hilbert construction itself, which is classical, but in the
fine--structural localisation of this representation mechanism.
The kernel--to--Hilbert passage emerges here as a canonical consequence of
bounded definability over \(J_2\) once a single aggregation principle is
fixed.
All objects entering the construction---local fragments, comparison data,
kernels, and the resulting Hilbert--space representation---are induced
canonically from this minimal input.

The subsequent sections analyse structural consequences of the resulting
representation.
These are not additional axioms or independent assumptions, but intrinsic
constraints forced by bounded definability and ultrafilter aggregation.
They serve to delimit the rigidity and expressive power of the induced
Hilbert--space structure rather than to introduce new analytic content.

The paper is organised as follows.
Section~\ref{sec:preliminaries} recalls background material on the Jensen
hierarchy and bounded definability.
Section~\ref{sec:fragments-ultrasheaves} introduces locally definable
fragments and their aggregation into Ultrasheaves.
Section~\ref{sec:hilbert} develops the induced comparison kernel and the
associated Hilbert--space representation.
Section~\ref{sec:consequences} records several structural consequences of
the construction.
Section~\ref{sec:conclusion} concludes with brief remarks.



%============================================================
\section{Definability in the Jensen Level $J_2$}
\label{sec:preliminaries}
%============================================================
We recall the background required for the subsequent construction and
fix notation.
The presentation is deliberately brief and restricted to standard facts
from the fine structure of the Jensen hierarchy and bounded
definability.

We work with the initial levels of Jensen’s hierarchy.
Let $J_0=\varnothing$, let $J_1=\mathrm{rud}(J_0)=\mathrm{HF}$ denote the
set of hereditarily finite sets, and let
\[
J_2=\mathrm{rud}(J_1)=\mathrm{rud}(\mathrm{HF}).
\]
The structure $J_2$ is a countable transitive set closed under all
rudimentary functions.
No further axiomatic assumptions are imposed.

Throughout the paper, definability over $J_2$ is understood exclusively
in terms of bounded formulas.
A subset $X\subseteq J_2$ is said to be \emph{$\Delta_0$--definable} if it
is defined by a bounded formula in the language of set theory, allowing
parameters from $J_2$.
Bounded definability is stable under rudimentary operations and provides
the basic notion of locality used in what follows.

In later sections we consider families of $\Delta_0$--definable subsets
of $J_2$ indexed by countable sets and aggregate them using ultrafilters.
This aggregation separates local definability from global structure.
While all local data are $\Delta_0$--definable subsets of $J_2$, the
resulting aggregated objects need not themselves belong to $J_2$.
This distinction underlies the Ultrasheaf construction and the
Hilbert--space representations developed below.

%============================================================
\section{Locally Definable Fragments and Ultrasheaves}
\label{sec:fragments-ultrasheaves}
%============================================================
We introduce the local objects used in the subsequent aggregation and
fix the corresponding equivalence notion.
All notions of locality are understood exclusively in terms of bounded
definability over the transitive Jensen level $J_2$.

\begin{definition}[Locally definable fragment]
A \emph{locally definable fragment} is a subset $X\subseteq J_2$ that is
$\Delta_0$--definable over the structure $(J_2,\in)$, allowing parameters
from $J_2$.
\end{definition}

\noindent
Locally $\Delta_0$--definable fragments form a class closed under the
rudimentary operations of $J_2$ and exactly capture bounded definability.

\begin{definition}[$\mathcal U$--equivalence]
Let $I$ be a countable index set and let $(X_i)_{i\in I}$ be a family of
locally definable fragments.
Fix a non--principal ultrafilter $\mathcal U$ on $I$.
Two families $(X_i)_{i\in I}$ and $(Y_i)_{i\in I}$ are said to be
\emph{$\mathcal U$--equivalent} if
\[
\{\, i\in I : X_i = Y_i \,\} \in \mathcal U .
\]
\end{definition}
\noindent
The ultrafilter \(\mathcal U\) is fixed once and for all as a
metatheoretic parameter.
It is used solely to form equivalence classes of countable families of
\(\Delta_0\)--definable fragments; no internal presentation of
\(\mathcal U\) over \((J_2,\in)\) is part of the framework.
Equivalently, \(\mathcal U\) functions as an external aggregation scheme
rather than as an object of the background structure.



\begin{definition}[Ultrasheaf]
An \emph{Ultrasheaf} is the $\mathcal U$--equivalence class of a family
$(X_i)_{i\in I}$ of locally definable fragments.
We denote such a class by $\langle X_i\rangle_{\mathcal U}$.
\end{definition}
\noindent
Ultrasheaves are obtained by ultrafilter aggregation of families of
locally $\Delta_0$--definable fragments.
They are, in general, not elements of $J_2$, but second--order definable
objects arising from the aggregation procedure.

Any property specified by a bounded $\Delta_0$–formula and verified
pointwise on a $\mathcal U$–large set of indices induces a
well–defined attribute of the associated Ultrasheaf.
Different choices of non--principal ultrafilters may yield non--isomorphic
Ultrasheaves; no canonical choice of ultrafilter is assumed.
Once a choice of $\mathcal U$ is fixed, the resulting class of
Ultrasheaves admits canonical comparison constructions, which will be
used in the subsequent sections.

%============================================================
\section{Canonical Hilbert Structure}
\label{sec:hilbert}
%============================================================
We show that the definability--theoretic structure introduced above
admits a canonical Hilbert--space representation in a precise
representation--theoretic sense.
All constructions rely exclusively on bounded definability and
ultrafilter aggregation.

Let $\mathcal F$ be the class of locally $\Delta_0$--definable subsets of
$J_2$.
Assume a locally $\Delta_0$--definable numerical comparison invariant
\[
\langle\cdot,\cdot\rangle:\mathcal F\times\mathcal F\to \mathbb D,
\]
where $\mathbb D$ denotes the dyadic rationals coded in $J_2$.
The choice of $\mathbb D$ is purely conventional and serves only to fix a
concrete $\Delta_0$--definable numerical codomain; any such domain would
suffice for the construction.

The definition of $\langle\cdot,\cdot\rangle$ may involve bounded
membership tests in the (possibly infinite) fragments
$X,Y\subseteq J_2$ and is stable under rudimentary constructions.
The invariant is assumed to be hermitian with respect to the involution
of the numerical structure.

Comparison invariants satisfying these definability and stability
conditions are standard in fine--structural analysis.
In particular, $\Delta_0$--definable numerical assignments over the
transitive structure $(J_2,\in)$ are closed under rudimentary
constructions; see, for example,~\cite{Devlin1984,Weaver2005}.
Throughout this section, one such invariant is fixed as background data.

\begin{definition}[Ultrasheaf comparison kernel]
Let $\langle\cdot,\cdot\rangle$ be fixed as above.
For Ultrasheaves $\mathfrak X=\langle X_i\rangle_{\mathcal U}$ and
$\mathfrak Y=\langle Y_i\rangle_{\mathcal U}$, define the
\emph{Ultrasheaf comparison kernel} $K$ by
\[
K(\mathfrak X,\mathfrak Y)
\;:=\;
\bigl[\, i \mapsto \langle X_i , Y_i \rangle \,\bigr]_{\mathcal U}.
\]
\end{definition}

\noindent
The kernel $K$ takes values in the space of $\mathcal U$--equivalence
classes of $\mathbb D$--valued families indexed by $I$.
It is independent of the choice of representatives by
$\Delta_0$--stability and invariance under $\mathcal U$--equivalence.
If the local comparison invariant $\langle\cdot,\cdot\rangle$ is
hermitian, then $K$ inherits the corresponding hermitian symmetry with
respect to the involution induced by pointwise conjugation on
$\mathbb D$.

\begin{lemma}[Finitary positivity condition]
\label{lem:finitary-positivity}
Assume that the local comparison invariant
$\langle\cdot,\cdot\rangle:\mathcal F\times\mathcal F\to\mathbb D$
satisfies the following finitary positivity condition.

\medskip
\noindent
For every $n<\omega$, every choice of locally $\Delta_0$--definable
families $(X_i^{(p)})_{i\in I}$ for $p<n$, and all coefficients
$c_0,\dots,c_{n-1}\in\mathbb D$, the quadratic form
\[
\sum_{p,q<n} \overline{c_p}\,
\langle X_i^{(p)}, X_i^{(q)}\rangle \, c_q
\]
is non--negative for $\mathcal U$--many indices $i\in I$.

\medskip
\noindent
Equivalently, all finite Gram matrices associated with the local
comparison invariant are positive semidefinite on $\mathcal U$--large
sets.
\end{lemma}

\begin{proof}[Sketch of proof]
Fix $n<\omega$, locally $\Delta_0$--definable families
$(X_i^{(p)})_{p<n}$, and coefficients $c_0,\dots,c_{n-1}\in\mathbb D$.
By the finitary positivity condition, the associated quadratic form is
non--negative on a $\mathcal U$--large set of indices.

For fixed $n$ and fixed families, the expression above depends only on
finitely many values of the local comparison invariant and is therefore
determined by a finite system of algebraic inequalities.
Since these inequalities are evaluated pointwise in $i$, their validity
on a $\mathcal U$--large set is preserved under ultrafilter aggregation.
\end{proof}

\noindent
The positivity of the local comparison invariant is a structural assumption.
It is not derived from definability considerations, but isolates the minimal
condition under which the aggregated comparison data admit a Hilbert--space
representation.


\begin{corollary}[Positive semidefiniteness of the comparison kernel]
\label{cor:kernel-positive}
Under the assumptions of Lemma~\ref{lem:finitary-positivity}, the
Ultrasheaf comparison kernel $K$ is positive semidefinite.
\end{corollary}

\begin{proof}
Fix Ultrasheaves $\mathfrak X_0,\dots,\mathfrak X_{n-1}$ with
representatives $\mathfrak X_p=\langle X_i^{(p)}\rangle_{\mathcal U}$.
Then for any fixed choice of coefficients $c_0,\dots,c_{n-1}\in\mathbb D$,
\[
\sum_{p,q<n} \overline{c_p}\,
K(\mathfrak X_p,\mathfrak X_q)\, c_q \ge 0,
\]
which is exactly the defining condition for positive semidefiniteness of
$K$.
\end{proof}

\medskip
\noindent
Given a positive semidefinite kernel $K$ on Ultrasheaves, taking values
in the \(\mathcal U\)--equivalence classes of numerical families, the
standard kernel construction yields a pre--Hilbert space $H_0$ and a
canonical mapping
\[
\kappa \colon \mathfrak X \longmapsto \kappa(\mathfrak X)\in H_0
\]
satisfying
\[
\langle \kappa(\mathfrak X), \kappa(\mathfrak Y)\rangle_{H_0}
\;=\;
K(\mathfrak X,\mathfrak Y),
\]
as in the classical theory of positive definite kernels; see, for
example, \cite{Aronszajn1950}.


\begin{proposition}[Canonical Hilbert representation]
\label{prop:prehilbert}
The metric completion of $H_0$ yields a Hilbert space $\mathcal H$ over
the numerical equivalence classes arising from ultrafilter aggregation,
uniquely determined up to isometric isomorphism by the kernel $K$.
\end{proposition}

\noindent
Here ``Hilbert space'' refers to the completion of the induced pre--Hilbert
space with respect to its inner product over the aggregated numerical
quotient.
Thus \(\mathcal H\) carries precisely the structure forced by the kernel:
an inner product and its metric completion.


\medskip
\noindent
The resulting Hilbert--space representation is canonical in the standard
representation--theoretic sense, that is, uniquely determined by the kernel
up to unitary equivalence.
Canonically means: the representation is determined by \(K\) uniquely up to
unitary equivalence.
Accordingly, no further choices enter the construction beyond those already
encoded by the kernel.


\medskip
\noindent
Identifying $H_0$ with its canonical image in $\mathcal H$, the map
$\kappa$ may be regarded as taking values in $\mathcal H$.
The space $\mathcal H$ is thus canonically determined, relative to the
chosen comparison invariant, as a representational completion of the
definability--theoretic data encoded by Ultrasheaves.

%============================================================
\section{Structural Consequences}
\label{sec:consequences}
%============================================================
This section records structural properties of the induced Hilbert--space
representation that are not assumed a priori, but follow solely from
bounded \(\Delta_0\)--definability, ultrafilter aggregation, and the
existence of a positive semidefinite comparison kernel.
No additional definability or structural assumptions are introduced.

The statements collected here isolate consequences that are intrinsic to
the representation mechanism itself.
They are not independent features added to the formalism, but unavoidable
structural constraints enforced by the kernel construction.
Wherever possible, results are formulated entirely at the level of the
comparison kernel; references to the associated Hilbert--space
representation serve only as a secondary interpretative layer.


\medskip
\noindent
\textbf{Quadratic structure.}
Let \(\mathfrak X=\langle X_i\rangle_{\mathcal U}\) be an Ultrasheaf, and
let \(\kappa(\mathfrak X)\in\mathcal H\) denote its image under the
canonical embedding induced by the comparison kernel.
The inner product on \(\mathcal H\) induces a canonical quadratic form
\[
\|\kappa(\mathfrak X)\|^2
\;:=\;
\langle \kappa(\mathfrak X),\kappa(\mathfrak X)\rangle_{\mathcal H}
\;=\;
K(\mathfrak X,\mathfrak X).
\]
This identity follows directly from the positive semidefiniteness of the
kernel and requires no additional assumptions.

\medskip
\noindent
\textbf{Canonical involution.}
The hermitian symmetry of the comparison kernel constructed in
Section~\ref{sec:hilbert} induces a canonical conjugation on the
associated Hilbert space \(\mathcal H\).
This involution is uniquely determined (up to unitary equivalence) by the
kernel and reflects symmetry properties of the underlying local comparison
invariants.
At the definability level, it is intrinsic to the aggregation procedure
and does not rely on any further structure.

\medskip
\noindent
\textbf{Factorisation and stabilisation.}
Factorisation properties of the comparison kernel are governed by
stabilisation phenomena at the level of bounded definability.

Let \((X_i)_{i\in I}\) and \((Y_i)_{i\in I}\) be families of locally
\(\Delta_0\)--definable fragments giving rise to Ultrasheaves
\(\mathfrak X=\langle X_i\rangle_{\mathcal U}\) and
\(\mathfrak Y=\langle Y_i\rangle_{\mathcal U}\).
Fix a \(\Delta_0\)--definability equivalence relation \(\equiv\) on the
class \(\mathcal F\) of locally definable fragments, identifying fragments
that are indistinguishable with respect to the bounded definability data
relevant for the comparison invariant.

We say that a family \((X_i)\) stabilises modulo \(\equiv\) if there exists
a \(\mathcal U\)--large set \(U\subseteq I\) such that all \(X_i\) with
\(i\in U\) lie in a single \(\equiv\)--class.
Independent stabilisation of \((X_i)\) and \((Y_i)\) modulo \(\equiv\)
yields factorisation of the comparison kernel; failure of such
independent stabilisation precludes factorisation.
This dichotomy depends only on bounded definability and ultrafilter
aggregation.

\medskip
\noindent
\textbf{Kernel--level characterisation.}
Factorisation admits an intrinsic formulation entirely at the kernel
level.
Suppose that the relevant definability data decomposes into two
independent components, labelled \(a\) and \(b\).
This induces corresponding kernel--level components of Ultrasheaves,
written
\[
\mathfrak X \longmapsto (\mathfrak X_a,\mathfrak X_b),
\qquad
\mathfrak Y \longmapsto (\mathfrak Y_a,\mathfrak Y_b),
\]
defined up to \(\mathcal U\)--equivalence.

An Ultrasheaf \(\mathfrak X\) is said to be \emph{separable} with respect
to this decomposition if, for all compatible Ultrasheaves
\(\mathfrak Y\),
\[
K(\mathfrak X,\mathfrak Y)
=
K_a(\mathfrak X_a,\mathfrak Y_a)\,
K_b(\mathfrak X_b,\mathfrak Y_b).
\]
Failure of such a factorisation indicates irreducible global coherence at
the kernel level.
All statements in this section are therefore purely kernel--theoretic and
do not presuppose any Hilbert--space representation.

\begin{definition}[Definability equivalence]
\label{def:definability-equivalence}
Let \(\mathcal F\) denote the class of locally \(\Delta_0\)--definable
subsets of \(J_2\), allowing parameters from \(J_2\).
A binary relation \(\equiv\) on \(\mathcal F\) is called a
\emph{\(\Delta_0\)--definability equivalence relation} if there exists a
bounded \(\Delta_0\)--formula \(\varphi(X,Y,\vec p)\) with parameters
\(\vec p\in J_2\) such that for all \(X,Y\in\mathcal F\),
\[
X \equiv Y
\quad\Longleftrightarrow\quad
(J_2,\in)\models \varphi(X,Y,\vec p).
\]
\end{definition}

\begin{lemma}[Independent stabilisation implies factorisation]
\label{lem:indep-stab-factor}
Let \((X_i)_{i\in I}\) and \((Y_i)_{i\in I}\) be families of locally
\(\Delta_0\)--definable fragments, and let
\(\mathfrak X=\langle X_i\rangle_{\mathcal U}\) and
\(\mathfrak Y=\langle Y_i\rangle_{\mathcal U}\) be the induced Ultrasheaves.
Assume that the local comparison invariant
\(\langle\cdot,\cdot\rangle:\mathcal F\times\mathcal F\to\mathbb D\)
is \(\Delta_0\)--definable and hermitian.

If there exist a \(\Delta_0\)--definability equivalence relation
\(\equiv\) and a \(\mathcal U\)--large set \(U\subseteq I\) such that both
\((X_i)\) and \((Y_i)\) stabilise modulo \(\equiv\) on \(U\), and such that
on \(U\)
\[
\langle X_i,Y_i\rangle
=
\alpha([X_i]_\equiv)\cdot \beta([Y_i]_\equiv)
\]
for functions \(\alpha,\beta\) depending only on \(\equiv\)--classes, then
the induced comparison kernel factorises.
\end{lemma}

\begin{proof}[Sketch]
Stabilisation implies that the \(\equiv\)--classes \([X_i]_\equiv\) and
\([Y_i]_\equiv\) are \(\mathcal U\)--eventually constant.
Since \(\alpha\) and \(\beta\) depend only on these classes, the associated
functions are \(\mathcal U\)--eventually constant as well.
Passing to ultrafilter equivalence classes yields
\[
K(\mathfrak X,\mathfrak Y)
=
\alpha(\mathfrak X)\,\beta(\mathfrak Y),
\]
establishing factorisation.
\end{proof}

\begin{corollary}[Factorisation versus definability dependence]
\label{cor:factorisation}
Let \(\mathfrak X\) and \(\mathfrak Y\) be Ultrasheaves induced by locally
\(\Delta_0\)--definable families.
The comparison kernel \(K(\mathfrak X,\mathfrak Y)\) factorises if and only
if, on a \(\mathcal U\)--large set of indices, the local comparison
invariant admits a multiplicative separation determined solely by the
stabilised definability classes of the representing families.
\end{corollary}


%============================================================
\section{Conclusion}
\label{sec:conclusion}
%============================================================
We have shown that Hilbert--space representations arise canonically from
bounded definability alone.
Starting from $\Delta_0$--definable local fragments over the Jensen level
$J_2=\mathrm{rud}(\mathrm{HF})$, ultrafilter aggregation yields global
second--order objects, termed Ultrasheaves.
Under ultrafilter aggregation, locally $\Delta_0$--definable numerical
comparison invariants give rise to positive semidefinite kernels.
The standard kernel construction then yields a canonical
Hilbert--space representation.

These results isolate a minimal structural core underlying
Hilbert--space theory.
Quadratic structure, factorisation and non--factorisation phenomena, and
the existence of a canonical conjugation arise as formal consequences of
bounded definability, ultrafilter aggregation, and positivity of the
induced kernel.
From this perspective, Hilbert space appears as a representational
completion of definability--theoretic data, uniquely determined (up to
isometric isomorphism) by the chosen comparison invariant.

From a logical perspective, the significance of the construction lies in
showing that bounded definability over a very low level of the Jensen
hierarchy already suffices to enforce a rigid representation--theoretic
geometry.
The conclusions are formal and structural; questions of interpretation or
dynamics lie outside the scope of the present note.



Several questions remain open.
These include a systematic analysis of classes of locally definable
comparison invariants, a finer investigation of the dependence of the
resulting Hilbert--space representations on the choice of ultrafilter,
and the identification of further structural features arising purely
from definability--theoretic considerations.
More broadly, the construction suggests a general representation--theoretic
principle in which coherent global structure emerges from bounded
definability via aggregation, inviting further study in other
definability--controlled settings.

%============================================================
\bibliographystyle{amsalpha}
\bibliography{ultrasheaf-logic}
%============================================================

\end{document}
